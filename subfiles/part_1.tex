\documentclass[../document.tex]{subfiles}


\begin{document}
    \section{Описание предметной области}
    \subsection{Машинное обучение}
    \par \acrfull{ml} -- это раздел искусственного интеллекта, изучающий методы построения алгоритмов, способных автоматически обучаться и улучшать свою работу на основе данных без явного программирования. В отличие от традиционных алгоритмов, где поведение системы жестко задаётся разработчиком, модели машинного обучения выявляют закономерности в данных и используют их для прогнозирования, классификации или принятия решений
    \par Результат обучения алгоритма называется моделью -- параметризированное отражение(функция), которое преобразует объекты из пространства входных признаков в пространство предсказаний. 
    \par Одним из главных требований к модели -- ее способность к обобщению. Благодаря этому модель не просто запомнит данные на которых училась, а находит в них закономерности, что позволяет более точно делать отражение на новых объектах.
    \par Алгоритмы \acrshort{ml} делятся на три основные категории:
    \begin{itemize}
        \item \nameref{sec:with_teacher}
        \item \nameref{sec:with_teacher}
        \item \nameref{sec:reinforcement_learning}
    \end{itemize} 

    \subfile{part_1/study_with_teacher.tex}

    \subfile{part_1/study_without_teacher.tex}

    \subfile{part_1/reinforcement_learning.tex}

    
    \subfile{part_1/deep_learning.tex}
    \subfile{part_1/nlp.tex}
    \subfile{part_1/codebert.tex}
    \subfile{part_1/dino.tex}
\end{document}