\documentclass[../document.tex]{subfiles}

\begin{document}
\section{Заключение}
\par В данной работе было исследовано извлечение признакового представления исходного кода с использованием методов обучения без учителя для последующего применения в downstream-задачах. Основной фокус был направлен на адаптацию подхода \acrshort{dino} для обработки программного кода, что привело к созданию модели Dino.
\par Несмотря на значительно меньший объем тренировочных данных по сравнению с CodeBERT, модель Dino продемонстрировала близкое качество в задачах классификации кода.
\par Если \acrshort{mlm} ориентированы на предсказание конкретных токенов, то \acrshort{dino} фокусируется на извлечении обобщённых признаковых представлений данных, что делает его более предпочтительным для выбранных задач, так как модель Dino учится извлекать признаки инвариантные к аугментациям.
\par Для улучшения стоит произвести обучение с большим количеством тренировочных данных и вычислительных блоков.
\par Модель Dino подтвердила свою жизнеспособность как альтернатива feature extraction моделям. Хотя она и не превосходит CodeBERT в абсолютных метриках, ее эффективность при малых данных открывает новые возможности для внедрения ИИ в разработку ПО.
\end{document}