\documentclass[../part_2.tex]{subfiles}

\begin{document}
\subsection{Требования к аппаратному обеспечению}
\par Для обучения нейронной сети требуется:
\begin{enumerate}
    \item Графический процессор (GPU) с высокой производительностью и тензерными ядрами. Рекомендуется использовать GPU от Nvidia не менее чем с 16 ГБ видеопамяти. Меньший объем памяти может существенно увеличить необходимое время для обучения модели.
    \item Хранилище данных которое обеспечит высокую скорость доступа к файлам. Рекомендуется использовать SSD.
    \item Оперативная память для хранения датасета. Рекомендуется 16 ГБ.
\end{enumerate}
\par Так же можно увеличить количество графических процессоров, так как код позволяет обучать модель на нескольких GPU одновременно.
\subsection{Требование к програмному обеспечению}
\par Для обучения нейронной сети требуется:
\begin{enumerate}
    \item Операционная система на базе Linux, в связи с тем что libuv не может работать на Windows.
    \item Язык программирования python 3.11 или выше.
    \item Библиотека глубокого обучения PyTorch
    \item Библиотека машинного обучения Scikit Learn
    \item Среда разработки: Pycharm, VSCode, Jupyter Notebook
    \item Паралельная вычислительная платформа и программный интерфейс CUDA
\end{enumerate}
\par Важно убедиться что программные средства совместимы между собой.
\end{document}