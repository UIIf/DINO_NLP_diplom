\documentclass[../part_2.tex]{subfiles}

\begin{document}
    \subsection{Описание данных}
    \par Для обучения модели было решено использовать код только на одном языке, таким образом уменьшится разнообразие в данных и модель сможет обучиться легче.
    \par Из-за распространенности было решено остановиться на языке программирования C, в связи с его популярностью и так как множество языков программирования C подобные.
    \par Тренировочный набор данных был собран с помощью репозитория CodeforcesYoink. Он предоставляет инструмент для автоматизированного сбора примеров кода с платформы codeforces. С помощью этого инструмента можно извлекать код участников по определенным параметрам, таким как язык программирования или идентификатору задания.
    \par Были собранны данные из 104 задач. Количество тренировочных данных: 36489
    \par Для оценки результата работы использованы данные из Project CodeNet\cite{puri2021codenetlargescaleaicode} от IBM. Это масштабный датасет, содержащий более 14 миллионов образцов кода на 55 языках программирования. Датасет включает разнообразные задачи, метаданные(например статус выполнения, время работы кода, потребление памяти).
    % TODO перефразировать
    \par Из этого набора данных было выбрано 48 задач, в которых было больше всего примеров кода на C и у которых статус выполнения принят.
\end{document}