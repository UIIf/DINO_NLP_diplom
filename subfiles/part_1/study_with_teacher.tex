\documentclass[../part_1.tex]{subfiles}

\begin{document}
\subsubsection{Обучение с учителем} % Обучение с учителем
    \label{sec:with_teacher}
    \par Обучение с учителем -- это вид машинного обучения, при котором модель обучается на примерах, где для каждого входного объекта известен правильный ответ. Цель такого метода обучения -- построение модели которая будет способна предсказать ответ для ранее не встречавшихся примеров с заданной точностью.
    \par Основное преимущество обучения с учителем:
    \begin{itemize}
        \item \textit{Интерпретируемость} -- модель работает с заранее определенным пространством выходных значений, так как разметка чаще составляется человеком.
        \item \textit{Интуитивная оценка качества} -- для оценки часто используется интуитивно понятные метрики такие как точность или средний квадрат ошибки. 
    \end{itemize}
    \par Основные недостатки обучения с учителем:
    \begin{itemize}
        \item \textit{Зависимость от качества входных данных} -- эффективность модели напрямую определяется качеством размеченных данных, процесс создания которых требует значительных временных и трудозатрат. Для сложных задач объём требуемых данных может возрастать экспоненциально, что создаёт существенные практические ограничения.
        \item \textit{Проблема переобучения} -- существует риск избыточной подгонки модели под особенности обучающей выборки -- в таком случае алгоритм начинает воспроизводить не только значимые закономерности, но и случайные шумы, что резко снижает его способность к обобщению на новых данных.
    \end{itemize}
    \par Пример моделей которые обучаются с помощью обучения с учителем:
    \begin{itemize}
        \item Линейная регрессия
        \item Дерево решений
        \item Метод опорных векторов
    \end{itemize}
    
\end{document}