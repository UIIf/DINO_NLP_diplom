\documentclass[../part_1.tex]{subfiles}

\begin{document}
\subsubsection{Обучение с учителем}
    \label{sec:with_teacher}
    \par Обучение с учителем -- это вид машинного обучения, при котором модель обучается на примерах, где для каждого входного объекта известен правильный ответ. Цель такого метода обучения -- построение модели, которая будет способна предсказать ответ для ранее не встречавшихся примеров с заданной точностью.
    \par Основные преимущества обучения с учителем:
    \begin{itemize}
        \item \textit{Интерпретируемость} -- в задачах обучения с учителем ответы обычно интерпретируемы для человека, поскольку разметка данных выполняется вручную. Это позволяет моделям обучаться на четко определенных примерах, где каждому входу соответствует однозначный, понятный целевой признак.
        \item \textit{Интуитивная оценка качества} -- для оценки часто используется интуитивно понятные метрики такие как точность или средний квадрат ошибки. 
        \item \textit{Простота} -- обучение с учителем представляет собой задачу аппроксимации таблично заданной функции, что делает его наиболее простым и формализуемым видом обучения.
    \end{itemize}
    \par Основные недостатки обучения с учителем:
    \begin{itemize}
        \item \textit{Наличие разметки} -- создание размеченных данных требует значительных временных затрат, а для узкоспециализированных областей может быть практически невозможна без привлечения экспертов.
        \item \textit{Зависимость от качества входных данных} -- эффективность модели напрямую определяется качеством размеченных данных, процесс создания которых требует значительных временных и трудозатрат. Для сложных задач объём требуемых данных может возрастать экспоненциально, что создаёт существенные практические ограничения.
        \item \textit{Проблема переобучения} -- существует риск избыточной подгонки модели под особенности обучающей выборки -- в таком случае алгоритм начинает воспроизводить не только значимые закономерности, но и случайные шумы, что резко снижает его способность к обобщению на новых данных.
    \end{itemize}
    \par К моделям, обучаемым данным методом, можно отнести такие как:
    \begin{itemize}
        \item Линейная регрессия\cite{linearregression}
        \item Дерево решений\cite{dessisiontree}
        \item Метод опорных векторов\cite{svm}
    \end{itemize}
    
\end{document}