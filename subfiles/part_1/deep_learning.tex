\documentclass[../part_1.tex]{subfiles}

\begin{document}
\subsection{Глубокое обучение}
\par Глубокое обучение -- это подраздел машинного обучения, основанный на использовании искусственных нейронных сетей (\acrshort{nn}). Эти модели способны автоматически извлекать признаки из данных, имитируя работу человеческого мозга в упрощённой форме.
\par Свое название оно получило благодаря многослойной архитектуре нейронных сетей, что позволяет моделям находить более сложные закономерности. Например, в задаче классификации изображений первый слой находит базовые признаки, такие как края и прямые линии. Второй слой используя уже обработанные данные находит углы. Таким образом при достаточном количестве слоев нейронная сеть сможет отличать котов от собак. 
\par К сожалению у нейронных сетей есть и недостатки:
\begin{itemize}
    \item \textit{Сложность интерпретации} -- так как \acrshort{nn} работает как "черный ящик" (мы знаем вход и выходи модели, но не знаем что происходит внутри).
    \item \textit{Вычислительная сложность} -- современные архитектуры могут иметь миллионы и миллиарды параметров, которые необходимо хранить в оперативной памяти.
\end{itemize}
\par В настоящее время глубокое обучение стало неотъемлемой частью нашей жизни. Благодаря ему работают голосовые помощники, системы рекомендаций и системы распознавания лиц.
\par Основные направления в которых применяют глубокое обучение:
\begin{itemize}
    \item \textit{Компьютерное зрение} -- Детекция объектов, классификация и генерация изображений.
    \item \textit{Обработка естественного языка} -- Машинный перевод, языковое моделирование.
    \item \textit{Обучение с подкреплением} -- Робототехника, Игровые AI.
\end{itemize}
\end{document}