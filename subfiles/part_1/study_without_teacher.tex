\documentclass[../part_1.tex]{subfiles}

\begin{document}
\subsubsection{Обучение без учителя} 
    \label{sec:without_teacher}
    \par Обучение без учителя --  это вид машинного обучения, при котором модель обучается на примерах, для которых известны только входные данные. В таких задачах, как правило, отсутствует верное решение, а качество результата оценивается исключительно с помощью специально разработанных метрик.
    \par Задачи обучения без учителя включают в себя:
    \begin{itemize}
        \item \textit{Кластеризацию} -- задача разделения набора данных на однородные группы таким образом, чтобы объекты внутри одного кластера были максимально схожи между собой, а объекты из разных кластеров -- максимально различны.
        \item \textit{Снижение размерности} -- задача уменьшения объема данных за счет выделения ключевой информации.
        \item \textit{Поиск ассоциативных правил} -- задача выявления устойчивых взаимосвязей между событиями в больших данных.
        \item \textit{Генеративные модели} -- задача создания новых объектов на основе некоторых входных данных, задающих параметры выходного объекта.
        \item \textit{Извлечение признаков} -- задача автоматического преобразования исходных данных в компактное и информативное векторое представление, пригодное для решения downstream задач.
    \end{itemize}
    \par Основные преимущества обучения без учителя:
    \begin{itemize}
        \item \textit{Не требуется разметка данных} -- работа с неразмеченными данными значительно облегчает процесс их сбора.
        \item \textit{Универсальность} -- применимо к большему классу задач, поскольку для обучения требуется лишь мера качества предсказания, которая может быть основанна на чем угодно, а не только на ответах.
    \end{itemize}
    \par Основные недостатки обучения без учителя:
    \begin{itemize}
        \item \textit{Сложность оценки качества} -- отсутствие разметки требует построеия не менее эффективной и объективной функции оценки качества предсказания.
        \item \textit{Проблема интерпретируемости} -- во многих задачах обучения без учителя предсказываемые моделью значения сложно интерпретируемы человеку за счет того, что оно опирается на некоторые функции, а не ответ человека.
    \end{itemize}
    \par К моделям, обучаемым данным методом, можно отнести такие как:
    \begin{itemize}
        \item KMeans\cite{kmeans}
        \item DBSCAN\cite{dbscan_book}
        \item GAN\cite{gan}
        \item LLM\cite{llm}
        \item Diffusion models\cite{song2021scorebasedgenerativemodelingstochastic}
        \item DINO
    \end{itemize}
\end{document}