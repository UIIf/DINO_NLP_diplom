\documentclass[../part_1.tex]{subfiles}

\begin{document}
\subsubsection{Обучение без учителя} % Обучение без учителем
    \label{sec:without_teacher}
    \par Обучение без учителя --  это вид машинного обучения, при котором модель обучается на примерах для которых нет какой-либо разметки. Цель такого метода обучения -- построение модели которая сама будет находить закономерности, не опираясь на внешние подсказки. 
    \par Задачи обучения без учителя включают в себя:
    \begin{itemize}
        \item \textbf{Кластеризацию} -- задача разделения объектов на группы, которые имеют сходство между собой и отличаются от других.
        \item \textbf{Снижение размерности} -- задача уменьшение количества признаков данных, сохраняя при этом информацию об объекте.
        \item \textbf{Поиск ассоциативных правил} -- задача выявления устойчивых взаимосвязей между событиями в больших данных.
        \item \textbf{Генеративные модели} -- задача генерации новых данных похожих на тренировочные.
    \end{itemize}
    \par Основное преимущество обучения без учителем:
    \begin{itemize}
        \item \textbf{Не требуется разметка данных} -- работа с неразмеченными данными значительно облегчает процесс сбора данных.
        \item \textbf{Гибкость и универсальность} -- применимо в разнообразных областях, а так же может использоваться для предобработки данных перед обучением с учителем
    \end{itemize}
    \par Основные недостатки обучения без учителем:
    \begin{itemize}
        \item \textbf{Сложность оценки качества} -- отсутсвие разметки затрудняет объективную оценку результатов.
        \item \textbf{Проблема интерпретируемости} -- так как пространство выходных значений не известно, сложно их интерпретировать. 
    \end{itemize}
    \par Пример модлей которые обучаются с помощью обучения с учителем:
    \begin{itemize}
        \item К средних
        \item Генеративные состязательные сети
    \end{itemize}
\end{document}