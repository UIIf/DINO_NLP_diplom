\documentclass[../part_1.tex]{subfiles}

\begin{document}
\subsubsection{Обучение без учителя} 
    \label{sec:without_teacher}
    % #TODO Зачастую в таких задачах не существует правильного ответа, а лишь известен метод оценки полученного результата
    \par Обучение без учителя --  это вид машинного обучения, при котором модель обучается на примерах для которых известны только входные данные. Цель такого метода обучения -- построение модели которая сама будет находить закономерности, не опираясь на внешние подсказки. 
    \par Задачи обучения без учителя включают в себя:
    \begin{itemize}
        % #TODO ПЕРЕПИСАТЬ, ВСМ ГРУППЫ МЕЖДУ СОБОЙ ПОХОЖИ
        \item \textit{Кластеризацию} -- задача разделения объектов на группы, которые имеют сходство между собой и отличаются от других.
        % #TODO Уменьшение объема данных об объекте за счет выделения ключевой информации
        \item \textit{Снижение размерности} -- задача уменьшение количества признаков данных, сохраняя при этом информацию об объекте.
        \item \textit{Поиск ассоциативных правил} -- задача выявления устойчивых взаимосвязей между событиями в больших данных.
        \item \textit{Генеративные модели} -- задача создания новых объектов на основе некоторых входных данных задающих параметры выходного объекта.
        % #TODO ИЗВЛЕЧЕНИЕ ПРИЗНАКОВ
    \end{itemize}
    \par Основные преимущества обучения без учителя:
    \begin{itemize}
        \item \textit{Не требуется разметка данных} -- работа с неразмеченными данными значительно облегчает процесс их сбора.
        % #TODO Применимо к большему классу задач, по скольку для обучения требуется лишь мера качества предсказания, которая может быть основанна на чем угодно а не только на ответах
        \item \textit{Гибкость и универсальность} -- применимо в разнообразных областях, а так же может использоваться для предобработки данных перед обучением с учителем.
    \end{itemize}
    \par Основные недостатки обучения без учителя:
    \begin{itemize}
        \item \textit{Сложность оценки качества} -- отсутствие разметки требует построеия не менее эффективной и объективной функции оценки качества предсказания.
        % #TODO ПОФИКСИТЬ
        \item \textit{Проблема интерпретируемости} -- во многих задачах обучения без учителя предсказываемые моделью значения сложно интерпретируемы человеку за счет того что оно опирается на некоторые функции а не ответ человека // из-за того что предсказание модели не похоже на нормальне ответы их интерпритация затрудняется. 
    \end{itemize}
    % #TODO К моделям обучаеммым данным методом можно отнести. Ссылка на литературу, добавить моделей
    \par Пример моделей которые обучаются с помощью обучения без учителем:
    \begin{itemize}
        \item KMeans
        \item CLUSTERING SMTH
        \item Генеративные состязательные сети
        \item LLM
        \item Diffusion models
        \item Feature extraction models
    \end{itemize}
\end{document}