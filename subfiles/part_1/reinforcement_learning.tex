\documentclass[../part_1.tex]{subfiles}

\begin{document}
\subsubsection{Обучение с подкреплением} % Обучение с уподкреплением
    \label{sec:reinforcement_learning}
    %  TODO переписать предложение
    \par Обучение с подкреплением --  это вид машинного обучения, при котором агент(модель) обучается на основе опыта взаимодействия со средой, принимая решения которые максимизируют награду.
    \par В отличие от прошлых методов, агенты ориентированы на последовательное принятие решений в условиях неопредленности.
    
    \par Основное преимущество обучения с подкреплением:
    \begin{itemize}
        \item \textit{Подходит для задач с отложенной наградой} -- может учитывать долгосрочные последствия действий, а не только мгновенную выгоду.
        \item \textit{Возможность обучения без размеченных данных} -- не требует готовых "правильных ответов".
        % #TODO Не требуется диффиренцируемая фкнеция качества
    \end{itemize}
    \par Основные недостатки обучения с подкреплением:
    \begin{itemize}
        % #TODO Переписать например ждать ответ человека для оценки ллм
        \item \textit{Высокие вычислительные затраты} -- такое обучение требует многочисленных операций взаимодействия со средой которые могут представлять собой трудновычислимые алгоритмы, например, игры.
        % #TODO Для получения качественной модели алгоритм обучения должен позволять модели исследовать новые способы решения задачи, в то время как эксплоатировать уже изученные способы. Пытаться найти новое решение, при этом портить чтобы найти новые
        \item \textit{Проблема исследования-эксплуатации} -- агент должен балансировать между исследованием и эксплуатацией. Исследование -- проба новых действий, для поиска лучшей стратегии. Эксплуатация -- использование уже известных лучших действий.
    \end{itemize}
    % #TODO Добавить литературу и моделей
    \par Пример моделей которые обучаются с помощью обучения с подкреплением:
    \begin{itemize}
        \item LLM
        \item Алгоритм автоматического управления .транспортных средства
    \end{itemize}
\end{document}