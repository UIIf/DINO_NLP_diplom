\documentclass[../part_3.tex]{subfiles}
\usepackage{multirow}
\begin{document}
\subsection{Сравнение моделей}
\par Сравнение было решено провести с моделями CodeBert и UnixCoder, так как они считаются лучшими на данный момент.
\par Изначально весь выбраный датасет был пропущен через модели DinoNLP, CodeBert и UnixCoder, а полученные вектора сохранены в numpy файлы.
\par В качестве базового решения (baseline) был использован наивный баесовский классификатор(Naive Bayes Classifier). Наивный байесовский классификатор был выбран в качестве baseline-модели, поскольку он является классическим методом машинного обучения, широко применяемым для задач обработки текстовых данных.  
\subsubsection{Классификация номера задания}
\par Было выбрано 48 заданий из набора данных от IBM в которых больше всего принятых решений. Размер тренировочного набора - 65059. Размер тестового набора - 16265. После этого было решено использовать KNN классификатор и MLP классификатор предсказания номера задачи.
\par Для KNN был выставлен параметр - 10 ближайших соседей.
\par MLP имело 11 слоев, с функциями активации ReLU.
\begin{table}[H]
    \centering
    \begin{tabular}{|c|c||c|c|c|c|}\hline 
        &BaseLine&&DinoNLP&CodeBert&UnixCoder\\ \hline 
        \multirow{2}{*}{Acc,\%}&\multirow{2}{*}{76}&KNN&80&80&89\\\cline{3-6}
        &&MLP&85&81&95\\\hline
    \end{tabular}  
    \caption{Точность в процентах решения задачи определения номера задания}
\end{table}
\par Решение при помощи модели DinoNLP значительно выше чем baseline, не уступает решению модели использующей представления модели CodeBert и хуже, чем UnixCoder.
\subsubsection{Классификация статуса задачи}
\par Из прошлого набора данных было выбрано одно задание в котором было больше всего примеров. В качестве целевой переменной было выбрано булевое значение, равен ли задача принятой. 
\par Размер тренировочного набора - 7838, из них 44\% приняты. Размер тестового набора - 1960, из них 44\% приняты. 
\begin{table}[H]
    \centering
    \begin{tabular}{|c|c||c|c|c|c|}\hline 
        &BaseLine&&DinoNLP&CodeBert&UnixCoder\\ \hline 
        \multirow{2}{*}{Acc,\%}&\multirow{2}{*}{72}&KNN&72&72&73\\\cline{3-6}
        &&MLP&77&79&85\\\hline
    \end{tabular}    
    \caption{Точность в процентах решения задачи определения статуса решения}
\end{table}
\par Решение при помощи модели DinoNLP значительно выше чем baseline, в некоторых случаях уступает решению модели использующей представления модели CodeBert и хуже, чем UnixCoder.
\end{document}