\documentclass{fefu}
\fefuloadstyle{imct}

\usepackage{subfiles}
\usepackage{amsmath}
\usepackage{mathtools}
\usepackage{amssymb}
\usepackage{nicefrac}
\usepackage{hyperref}
\usepackage{minted}
\usepackage{multirow}
\usepackage{subcaption}
\usepackage{placeins}
\addbibresource{references.bib}
\graphicspath{{./images/}}

\newcommand{\pa}[1]{\left(#1\right)}
\newcommand{\code}[2][text]{\mintinline{#1}{#2}}
\newcommand{\paramrow}[4][-]{\code{#2} & \code{#3} & \code{#1} & #4\\}
\newcommand{\paramrowlistvalue}[4][]{\code{#2} & \code{#3} & \code{[#1]} & #4\\}
\newcommand{\niceref}[2]{\hyperref[#1]{#2 \ref{#1}}}
\DeclareMathOperator{\sech}{sech}

\author{Петров Сергей Дмитриевич}
\setgroup{М9119-09.04.01иибд}
\setfaculty{09.04.01 Информатика и вычислительная техника}
\setprogram{Искусственный интеллект и большие данные}
\setsupervisor{Тыщено Андрей Геннадьевич}{ } %TODO
\setdirector{Мирин Илья Геннадьевич}{}%TODO
\setdeputy{Сапрыкина Елена Валерьевна}{к.э.н.}%TODO
\setreviewer{Луньков Андрей Аллександрович}{зав. лаб.  НЦВИ ИОФ РАН, к.ф.-м.н.}%TODO
\setexportsupervisor{Сапрыкина Елена Валерьевна}{зам. директора ШЦЭ}%TODO
\setsecretary{Тихонова Татьяна Сергеевна}{}%TODO
\title{Извлечение признакового представления исходного кода с использованием методов обучения без учителя для downstream обучения моделей}
\setconsultant{Тыщено Андрей Геннадьевич}%TODO


\date{07.07.21}

% \disablehyphenation %OPTIONAL

\makeglossaries

% \newacronym{pml}{PML}{\hyperref[sec::PML]{Perfectly Matched Layers}}
% \newacronym{ssp}{SSP}{\hyperref[sec::root_ssp]{Split-Step Pad\'e}}
% \newacronym{sel}{SEL}{\hyperref[sec::SEL]{Sound Exposure Level}}
% \newacronym{mpe}{МПУ}{Модовое Параболическое Уравнение}
% \newacronym{wampe}{ШМПУ}{Широкоугольное Модовое Параболическое Уравнение}

\begin{document}

    \maketitle{thesis}
    \makethesistitlebackside[1.2]
    \title{Извлечение признакового представления исходного кода с использованием \hfill\null\\ методов обучения без учителя для downstream обучения моделей\hfill\null}
    \subfile{subfiles/annotation.tex}
    \newpage
    \setcounter{page}{4}
    % \subfile{subfiles/abstract.tex}
    \tableofcontents
    % \subfile{subfiles/introduction.tex}
    \newpage
    % \printglossary[type=\acronymtype,title={Глоссарий}]
    \newpage
    % \subfile{subfiles/subject.tex}
    \newpage
    % \subfile{subfiles/math.tex}
    \newpage
    % \subfile{subfiles/program.tex}
    \newpage
    % \subfile{subfiles/conclusion.tex}
    \newpage
    \printbibliography
    \newpage
    % \subfile{subfiles/appendices.tex}
\end{document}